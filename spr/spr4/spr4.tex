\documentclass[a4paper,10pt]{article}
\usepackage[utf8]{inputenc}
\usepackage[T1]{fontenc}
\usepackage[polish]{babel}
\usepackage{graphicx}
\usepackage{mathtools}

%opening
\title{Badanie różnych implementacji tablic asocjacyjnej pod kątem dostępu
do jej elementów}
\author{Szymon Leśniak}

\begin{document}

\maketitle

\section{Wstęp}

\par Tablica asocjacyjna jest strukturą danych, w który zbiera pary danych, 
niekoniecznie tych samych typów: klucza, który zwykle jest wartością stałą, 
oraz wartości, dowolnie modyfikowalnej. Tablice asocjacyjne implementowane
są tak, by minimalizować czas dostępu (modyfikacji lub pobrania) wartości.
Jest to operacja na tyle popularna, że C++ STL definiuje ją za pomocą operatora
\verb+[]+ (szablon \verb+std::map+). 

\section{Opis badania i szczegóły implementacji}

\par Do badania przeznaczonych było 5 zestawów niepowtarzających się napisów
w różnej ilości: 100, 1000, 10000, 50000 i 100000 sztuk. Każdym z zestawów
zapełniano tablicę asocjacyjną. Mierzony był czas wpisania wartości do losowo
wybranego elementu tablicy. Operację powtarzano wielokrotnie w celu zniwelowania
błędów przypadkowych.

\par Badanymi implementacjami były: prosta tablica dynamiczna, drzewo binarne 
(bez mechanizmów równoważenia długości gałęzi) oraz tablicę haszującą. Funkcja
haszująca była oparta na operacji modulo. Ostatnia implementacja była badana w 
trzech wariantach, odpowiadających różnemu zapełnieniu tablicy: 10\%, 50\% i
90\%. Konieczne było zatem odejście od spotykanego zalecenia, by rozmiar 
wyrażał się liczbą pierwszą.

\section{Wyniki badania}

\par Wyniki są przedstawione w poniższej tabeli oraz na wykresie (skala 
log-log):

\begin{center}
\begin{table}[!h]

\caption{Czasy dostępu do elementów tablicy asocjacyjnej (w sekundach)}

\begin{tabular}{p{0.15\linewidth}|p{0.15\linewidth}|p{0.15\linewidth}|p{0.15\linewidth}|p{0.15\linewidth}|p{0.15\linewidth}}
\textbf{Ilość próbek} & \textbf{tradycyjna tablica} & \textbf{drzewo binarne} & \textbf{tablica haszująca (zapełnienie 10\%)} & \textbf{tablica haszująca (zapełnienie 50\%)} & \textbf{tablica haszująca (zapełnienie 90\%)}\\
\hline	
100 & 2,67\(\cdot 10^{-6}\) & 6,50\(\cdot 10^{-6}\) & 4,41\(\cdot 10^{-7}\) & 5,31\(\cdot 10^{-7}\) & 6,30\(\cdot 10^{-7}\)\\
1000 & 1,27\(\cdot 10^{-5}\) & 6,82\(\cdot 10^{-6}\) & 4,52\(\cdot 10^{-7}\) & 6,34\(\cdot 10^{-7}\) & 6,14\(\cdot 10^{-7}\)\\
10000 & 0,000452823 & 7,17\(\cdot 10^{-6}\) & 7,17\(\cdot 10^{-7}\) & 9,45\(\cdot 10^{-7}\) & 8,67\(\cdot 10^{-7}\)\\
50000 & 0,0126578 & 7,84\(\cdot 10^{-6}\) & 7,84\(\cdot 10^{-7}\) & 6,32\(\cdot 10^{-7}\) & 5,20\(\cdot 10^{-7}\)\\
100000 &  & 8,29\(\cdot 10^{-6}\) & 8,29\(\cdot 10^{-7}\) & 1,70\(\cdot 10^{-6}\) & 8,26\(\cdot 10^{-7}\)
\end{tabular}
\end{table}

\end{center}

\begin{figure}[!h]
\includegraphics[width=\linewidth]{dane.eps}
 
\end{figure}

\section{Wnioski}

\par Zgodnie z oczekiwaniami zwykła tablica dała najgorszy czas dostępu do danych.
Dostęp ma złożoność \(O(n)\) i jako taki jest nieakceptowalny dla większych 
zbiorów danych.

\par Potwierdzona została przewidwana złożoność dostępu \(O(1)\) dla tablicy
haszującej. Duży rozmiar tablicy przyspiesza nieznacznie średni czas dostępu do
danych, co wynika z mniejszego prawdopodobieństwa wystąpienia kolizji.

\par Jedynie dla drzewa binarnego zgodność z założeniami teoretycznymi (tj.
złożoność \(O(\log n)\)) jest widoczna w niewielkim stopniu.

\end{document}
