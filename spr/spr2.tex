\documentclass[a4paper,10pt]{article}
\usepackage[utf8]{inputenc}
\usepackage[T1]{fontenc}
\usepackage[polish]{babel}
\usepackage{graphicx}
\usepackage{mathtools}

%opening
\title{Badanie złożoności obliczeniowej algorytmów sortowania szybkiego
oraz przez scalanie}
\author{Szymon Leśniak}

\begin{document}

\maketitle

\section{Wstęp}

\par Sortowanie przez scalanie (\textit{merge sort}) opiera się na 
rekurencyjnym rozłożeniu sortowanej tablicy aż do pojedynczych elementów. 
Następnie dochodzi do scalenia tablic, już w uporządkowany sposób. Algorytm zarówno w średnim,
jak i w najgorszym przypadku ma złożoność \(O(n \log n)\).

\par W algorytmie sortowania szybkiego (\textit{quicksort}) wybiera się jeden
z elementów tablicy (nazywany \textit{pivot}), który jest umieszczany 
na właściwe jemu miejsce. Na powstałych po jego bokach podtablicach wywołuje
się rekurencyjnie ten algorytm. W średnim przypadku jego złożoność wynosi 
\(O(n \log n)\), jednak działa szybciej od innych algorytmów tego samego rzędu.
Algorytm działa najwolniej, gdy pivotem każdorazowo jest największa lub 
najmniejsza liczba z tablicy. Wówczas mamy \(O(n^2)\).

\section{Opis badania}

\par Sortowanym pojemnikiem użytym w badaniu był \verb+vector+ liczb typu
\verb+int+, pochodzący z STL C++. Implementacji 
dokonano przy użyciu systemu operacyjnego Ubuntu 13.10 oraz 
kompilatora g++ 4.8.1.

\par Sortowane tablice były wczytywane z pliku oraz porównywane ze wzorcem
zawartym w drugim pliku. W plikach wykorzystanych do badania każdy 
element tablicy z osobna był losowany, co powinno zapewnić realizację
przypadku średniego sortowania. Ostateczny wynik badania był średnią
z czasu wielokrotnego sortowania tej samej tablicy.

\section{Wyniki badania}

\par Wyniki są przedstawione w poniższej tabeli oraz na wykresie (skala 
log-log) 

\begin{center}
\begin{table}[h]
\caption{Wyniki działania algorytmów sortowania}
\begin{tabular}{llll}
\textbf{Ilość liczb} & \textbf{Ilość powtórzeń} & \textbf{MergeSort} & \textbf{QuickSort}\\
10 & 500 & 2,51\(\cdot 10^{-5}\) & 1,84\(\cdot 10^{-6}\)\\
100 & 100 & 0,000165963 & 1,36\(\cdot 10^{-5}\)\\
1000 & 50 & 0,00183497 & 0,000194129\\
10000 & 25 & 0,02109 & 0,00222775\\
100000 & 15 & 0,231004 & 0,0270007\\
1000000 & 10 & 2,54498 & 0,318073
\end{tabular}

\end{table}
\end{center}

\begin{figure}[h]
 \includegraphics[width=\textwidth]{sort.eps}
\end{figure}

\section{Wnioski}

\par W przypadku obu algorytmów widać niewielkie odchylenie od liniowej
złożości obliczeniowej -- należy więc się domyślać, że spełnione jest
oczekiwanie, że w obu przypadkach mamy \(O(n \log n)\).

\par Widoczne też stało się to, że \(O\)-notacja ignoruje czynniki
stałe -- nie daje informacji o tym, że MergeSort jest wyraźnie
wolniejszy.

\end{document}
